\documentclass[11pt, a4paper]{memoir}


\usepackage[utf8]{inputenc}
\usepackage[spanish]{babel}



\usepackage{amsmath,amsfonts,amssymb,amsthm}
\usepackage{booktabs}
\usepackage{hyperref}
\usepackage{authblk}

\usepackage[acronym]{glossaries}

\renewcommand\Affilfont{\itshape\small}
\renewcommand\Authand{ y }

\newtheorem{thm}{Teorema}

\usepackage{array}
\newcolumntype{L}[1]{>{\raggedright\let\newline\\\arraybackslash\hspace{0pt}}p{#1}}
\newcolumntype{C}[1]{>{\centering\let\newline\\\arraybackslash\hspace{0pt}}p{#1}}
\newcolumntype{R}[1]{>{\raggedleft\let\newline\\\arraybackslash\hspace{0pt}}p{#1}}

\newcolumntype{X}[1]{>{\raggedright\let\newline\\\arraybackslash\hspace{0pt}}m{#1}}
\newcolumntype{Y}[1]{>{\centering\let\newline\\\arraybackslash\hspace{0pt}}m{#1}}
\newcolumntype{Z}[1]{>{\raggedleft\let\newline\\\arraybackslash\hspace{0pt}}m{#1}}

\usepackage{color}
\makeglossaries

\newacronym{ndtm}{NDTM}{Máquina de Turing No Determinista}

\newacronym{sat}{SAT}{SATISFACTIBILIDAD}

\newacronym{3dm}{3DM}{3-DimentionalMatching}


\title{\Huge Seis problemas básicos $\mathcal{NP}$-completos}

\author{Luz Marina Moreno de Antonio}
\author{Jorge Riera Ledesma}

\affil{Departamento de Ingeniería Informática y de Sistemas. Universidad de La Laguna}

\begin{document}
\maketitle

\chapter{3SAT}

\section{Problemas involucrados}

\subsection*{\gls{sat}}

\noindent ENTRADA: Un conjunto de cláusulas $C=\left \{c_1, c_2, \dots, c_m \right \}$ sobre un conjunto finito $U$ de variables.

\noindent PREGUNTA: ¿Existe una asignación booleana para $U$, tal que satisfaga todas las cláusulas de $C$? 


\subsection*{\gls{3sat}}

\noindent ENTRADA: Un conjunto de cláusulas $C=\left \{c_1, c_2, \dots, c_m \right \}$ sobre un conjunto finito $U$ de variables tal que $|c_i|=3$, para $1\le i \le m$.

\noindent PREGUNTA: ¿Existe una asignación booleana para $U$, tal que satisfaga todas las cláusulas de $C$? 

\section{Demostración de NP-completitud}

\begin{thm}
	\gls{3sat} es $\mathcal{NP}$-completo.
\end{thm}

\begin{proof}
Es fácil comprobar que \gls{3sat}	$\in \mathcal{NP}$, ya que se puede encontrar una algoritmo para una \gls{ndtm} que reconozca el lenguaje $L(\mbox{3SAT},e)$, para un esquema de codificación $e$, en un número de pasos acotado por una función polinomial.

Transformaremos \gls{sat} en \gls{3sat}. Sea $U=\left \{u_1, u_2, \dots, u_n  \right\}$ un conjunto de variables, y $C=\left \{c_1, c_2, \dots, c_m \right \}$ un conjunto de cláusulas conformando una entrada arbitraria de \gls{sat}. Construiremos una conjunto $C'$ de cláusulas de tres literales, basadas en un conjunto $U'$ de variables, tal que $C'$ es satisfactible si y solo si $C$ es satisfactible.

La construcción de $C'$ simplemente remplazará cada cláusula individual $c_j\in C$ por un conjunto equivalente $C'_j$ de cláusulas de tres literales, basada en las variables originales $U$, y algunas variables adicionales $U'_j$, cuyo uso estará restringido a las cláusulas de $C'_j$. Esta construcción dará lugar a los conjuntos 
\[
U' = U \cup \left( \bigcup_{j=1}^m U'_j\right)
\]
y
\[
C' = \bigcup_{j=1}^mC_j.
\]
De esta manera, sólo se necesita demostrar cómo construir $C'_j$ y $U'_j$ a partir de $c_j$.

Supóngase que la cláusula $c_j$ viene definida por los literales $\left \{ z_1, z_2, \dots, z_k  \right\}$, donde cada uno de estos elementos deriva de las variables de $U$. La forma en que construimos los conjuntos  $C'_j$ y $U'_j$ va a depender del valor de $k$.

\vspace{0.5cm}

\begin{tabular}{@{}lll@{}}
Caso 1. & $k = 1$. & $U'_j = \left \{ y_j^1, y_j^2 \right \}$ \\
        &          &$C'_j = \left \{ \left \{  z_1, y_j^1, y_j^2  \right \}, \left \{  z_1, \bar{y}_j^1, y_j^2  \right \} , \left \{  z_1, y_j^1, \bar{y}_j^2  \right \}, \left \{  z_1, \bar{y}_j^1, \bar{y}_j^2  \right \}   \right \}$ 	\\
Caso 2. & $k = 2$. & $U'_j = \left \{ y_j^1 \right \}$ \\
        &          &$C'_j = \left \{ \left \{  z_1, z_2, y_j^1  \right \}, \left \{  z_1, z_2, \bar{y}_j^1  \right \}   \right \}$ 	\\
Caso 3. & $k = 3$. & $U'_j = \emptyset$ \\
        &          &$C'_j = \left \{  c_j  \right \}$ 	\\        
Caso 4. & $k = 4$. & $U'_j = \left \{ y^i_j | 1 \le i \le k - 3  \right \}$ \\
        &          &$C'_j = \left \{  z_1, z_2, y_j^1  \right \} \cup \left\{\left \{ \bar{y}_j^i, z_{i + 2}, y_j^{i + 1}  \right \} | 1 \le i \le k - 4 \right \} \cup \left \{ \bar{y}_j^{k -3}, z_{k -1}, z_k  \right \}$ 	\\
\end{tabular}

\vspace{0.5cm}

Para probar que esto es en efecto una transformación, debemos demostrar que el conjunto de cláusulas $C'$ es satisfactible si y sólo si el conjunto $C$ lo es. Supóngase primero que $t:U\rightarrow \left \{T,F \right \}$ es una asignación booleana que satisface $C$. Veremos a continuación que $t$ puede ser extendida a una asignación booleana $t':U'\rightarrow \left \{T,F \right \}$ satisfaciendo $C'$. Puesto que las variables $U' - U$ están particionadas en conjuntos $U'_j$, y puesto que las variables de cada conjunto $U'_j$ sólo aparecen en las cláusulas $C'_j$, debemos demostrar cómo $t$ puede extenderse para cada conjunto $U'_j$ de forma independiente, en cada caso sólo tenemos que verificar que las cláusulas de $C'_j$ son satisfechas. Haremos esto de la siguiente manera:

\begin{itemize}
	\item Si se ha construido $U'_j$ sobre los casos 1 o 2, entonces las cláusulas de $C'_j$ ya son satisfechas por $t$, de manera que podemos extender arbitrariamente a $U'_j$, por ejemplo, asignando $t'(y) = T$, para todo $y\in U'_j$. 
	\item Si $U'$ se ha construido sobre el caso 3 $U'_j$ es vacío, y por tanto, toda cláusula de $C'$ ya está satisfecha por $t$.
	\item Si se ha construido sobre el caso 4 la cláusula $c_j=\left \{ z_1, z_2, \dots, z_k \right \}$, con $k > 3$. Como $t$ es una asignación booleana que satisface $C$, entonces debe haber un entero $l$ tal que el literal $z_l$ tiene un valor verdadero mediante $t$.  
	\begin{itemize} 
	\item Si $l$ es 1 o 2, entonces se asignará $t'(y^i_j) = F$ para $1 \le i \le k - 3$.
	\item Si $l$ es $k - 1$ o $k$, entonces se asignará  $t'(y^i_j) = T$  para $1 \le i \le k - 3$.
	\item En cualquier otro caso, se asignará  $t'(y^i_j) = T$  para $1 \le i \le l - 2$, y $t'(y^i_j) = F$  para $l -1 \le i \le k - 3$.
	\end{itemize} 
\end{itemize}
Es fácil verificar que estas opciones garantizan que todas las cláusulas de $C'_j$ son satisfechas, y por lo tanto todas las de $C'$ mediante $t'$.  

En el otro sentido, si $t'$ es una asignación booleana que satisface $C'$, es fácil verificar que la restricción de $t'$ a las variables de $U$ debe también satisfacer $C$. Entonces, $C'$ es satisfactible si y sólo si $C$ también lo es.

Para comprobar que esta transformación puede llevarse a cabo en tiempo polinomial es suficiente con observar que el número de cláusulas de tres literales de $C'$ está acotada por un polinomio en $nm$. Por lo tanto, el tamaño de cada entrada de \gls{3sat} está acotado superiormente por una función polinómica del tamaño de la entrada de \gls{sat}. 

\end{proof}


\clearpage
\printglossary[type=\acronymtype]

\end{document}
